\PassOptionsToPackage{unicode=true}{hyperref} % options for packages loaded elsewhere
\PassOptionsToPackage{hyphens}{url}
%
\documentclass[ignorenonframetext,]{beamer}
\usepackage{pgfpages}
\setbeamertemplate{caption}[numbered]
\setbeamertemplate{caption label separator}{: }
\setbeamercolor{caption name}{fg=normal text.fg}
\beamertemplatenavigationsymbolsempty
% Prevent slide breaks in the middle of a paragraph:
\widowpenalties 1 10000
\raggedbottom
\setbeamertemplate{part page}{
\centering
\begin{beamercolorbox}[sep=16pt,center]{part title}
  \usebeamerfont{part title}\insertpart\par
\end{beamercolorbox}
}
\setbeamertemplate{section page}{
\centering
\begin{beamercolorbox}[sep=12pt,center]{part title}
  \usebeamerfont{section title}\insertsection\par
\end{beamercolorbox}
}
\setbeamertemplate{subsection page}{
\centering
\begin{beamercolorbox}[sep=8pt,center]{part title}
  \usebeamerfont{subsection title}\insertsubsection\par
\end{beamercolorbox}
}
\AtBeginPart{
  \frame{\partpage}
}
\AtBeginSection{
  \ifbibliography
  \else
    \frame{\sectionpage}
  \fi
}
\AtBeginSubsection{
  \frame{\subsectionpage}
}
\usepackage{lmodern}
\usepackage{amssymb,amsmath}
\usepackage{ifxetex,ifluatex}
\usepackage{fixltx2e} % provides \textsubscript
\ifnum 0\ifxetex 1\fi\ifluatex 1\fi=0 % if pdftex
  \usepackage[T1]{fontenc}
  \usepackage[utf8]{inputenc}
  \usepackage{textcomp} % provides euro and other symbols
\else % if luatex or xelatex
  \usepackage{unicode-math}
  \defaultfontfeatures{Ligatures=TeX,Scale=MatchLowercase}
\fi
\usetheme[]{Frankfurt}
\usecolortheme{dolphin}
% use upquote if available, for straight quotes in verbatim environments
\IfFileExists{upquote.sty}{\usepackage{upquote}}{}
% use microtype if available
\IfFileExists{microtype.sty}{%
\usepackage[]{microtype}
\UseMicrotypeSet[protrusion]{basicmath} % disable protrusion for tt fonts
}{}
\IfFileExists{parskip.sty}{%
\usepackage{parskip}
}{% else
\setlength{\parindent}{0pt}
\setlength{\parskip}{6pt plus 2pt minus 1pt}
}
\usepackage{hyperref}
\hypersetup{
            pdftitle={Tema 3},
            pdfauthor={Jose María Hernández de la Cruz},
            pdfborder={0 0 0},
            breaklinks=true}
\urlstyle{same}  % don't use monospace font for urls
\newif\ifbibliography
\usepackage{longtable,booktabs}
\usepackage{caption}
% These lines are needed to make table captions work with longtable:
\makeatletter
\def\fnum@table{\tablename~\thetable}
\makeatother
\setlength{\emergencystretch}{3em}  % prevent overfull lines
\providecommand{\tightlist}{%
  \setlength{\itemsep}{0pt}\setlength{\parskip}{0pt}}
\setcounter{secnumdepth}{0}

% set default figure placement to htbp
\makeatletter
\def\fps@figure{htbp}
\makeatother


\title{Tema 3}
\providecommand{\subtitle}[1]{}
\subtitle{Modalidad oracional}
\author{Jose María Hernández de la Cruz}
\date{20 de mayo de 2021}

\begin{document}
\frame{\titlepage}

\begin{frame}{Contenido}
\protect\hypertarget{contenido}{}

\tableofcontents

\end{frame}

\hypertarget{modalidad-oracional}{%
\section{Modalidad oracional}\label{modalidad-oracional}}

\begin{frame}{La modalidad oracional}
\protect\hypertarget{la-modalidad-oracional}{}

\begin{block}{Definición}

Dependiendo de la intención del hablante y de la actitud que tome ante
lo que dice, este utilizará diferentes \textbf{tipos de oraciones}.
Todas estas oraciones pueden ser \textbf{afirmativas} o
\textbf{negativas}:

\end{block}

\end{frame}

\begin{frame}{La modalidad oracional}
\protect\hypertarget{la-modalidad-oracional-1}{}

\begin{block}{Tabla}

\begin{longtable}[]{@{}llll@{}}
\toprule
\begin{minipage}[b]{0.15\columnwidth}\raggedright
\textbf{Modalidad}\strut
\end{minipage} & \begin{minipage}[b]{0.14\columnwidth}\raggedright
\textbf{Finalidad}\strut
\end{minipage} & \begin{minipage}[b]{0.18\columnwidth}\raggedright
\textbf{Entonacion}\strut
\end{minipage} & \begin{minipage}[b]{0.41\columnwidth}\raggedright
\textbf{Ejemplos}\strut
\end{minipage}\tabularnewline
\midrule
\endhead
\begin{minipage}[t]{0.15\columnwidth}\raggedright
Enunciativa\strut
\end{minipage} & \begin{minipage}[t]{0.14\columnwidth}\raggedright
Informan\strut
\end{minipage} & \begin{minipage}[t]{0.18\columnwidth}\raggedright
Neutra\strut
\end{minipage} & \begin{minipage}[t]{0.41\columnwidth}\raggedright
\emph{Son tus amigas}\strut
\end{minipage}\tabularnewline
\begin{minipage}[t]{0.15\columnwidth}\raggedright
Interrogativa\strut
\end{minipage} & \begin{minipage}[t]{0.14\columnwidth}\raggedright
Preguntan\strut
\end{minipage} & \begin{minipage}[t]{0.18\columnwidth}\raggedright
Interrogativa\strut
\end{minipage} & \begin{minipage}[t]{0.41\columnwidth}\raggedright
\emph{¿Son tus amigas?}\strut
\end{minipage}\tabularnewline
\begin{minipage}[t]{0.15\columnwidth}\raggedright
Exclamativa\strut
\end{minipage} & \begin{minipage}[t]{0.14\columnwidth}\raggedright
Enfatizan\strut
\end{minipage} & \begin{minipage}[t]{0.18\columnwidth}\raggedright
Exclamativa\strut
\end{minipage} & \begin{minipage}[t]{0.41\columnwidth}\raggedright
\emph{¡Son tus amigas!}\strut
\end{minipage}\tabularnewline
\begin{minipage}[t]{0.15\columnwidth}\raggedright
Dubitativa\strut
\end{minipage} & \begin{minipage}[t]{0.14\columnwidth}\raggedright
Dudan\strut
\end{minipage} & \begin{minipage}[t]{0.18\columnwidth}\raggedright
En suspensión\strut
\end{minipage} & \begin{minipage}[t]{0.41\columnwidth}\raggedright
\emph{Quizá sean tus amigas}\strut
\end{minipage}\tabularnewline
\begin{minipage}[t]{0.15\columnwidth}\raggedright
Desiderativa\strut
\end{minipage} & \begin{minipage}[t]{0.14\columnwidth}\raggedright
Desean\strut
\end{minipage} & \begin{minipage}[t]{0.18\columnwidth}\raggedright
Casi exclamativa\strut
\end{minipage} & \begin{minipage}[t]{0.41\columnwidth}\raggedright
\emph{Ojalá vengan tus amigas}\strut
\end{minipage}\tabularnewline
\begin{minipage}[t]{0.15\columnwidth}\raggedright
Exhortativa\strut
\end{minipage} & \begin{minipage}[t]{0.14\columnwidth}\raggedright
Ordenan o ruegan\strut
\end{minipage} & \begin{minipage}[t]{0.18\columnwidth}\raggedright
Exclamativa o neutra\strut
\end{minipage} & \begin{minipage}[t]{0.41\columnwidth}\raggedright
\emph{Ven con tus amigas.} \emph{Puedes venir con tus amigas.}\strut
\end{minipage}\tabularnewline
\bottomrule
\end{longtable}

\end{block}

\begin{block}{Explicación}

\begin{itemize}
\tightlist
\item
  \textbf{Enunciativas}. Presentan la acción sin que el hablante exprese
  ninguna actitud. Pueden ser afirmativas o negativas.
\item
  \textbf{Exclamativas}. Expresan sorpresa y admiración.
\item
  \textbf{Exhortativas}. Son aquellas que expresan una orden, mandato o
  recomendación. El verbo de estas oraciones suele ir en imperativo.
\item
  \textbf{Interrogativas}. Son aquellas que expresan una pregunta.
  Recuerda que puede haber interrogativas directas (llevan signos de
  interrogación al principio y al final de la misma) e indirectas
  (tienen sentido de pregunta, pero sin signos de interrogación)
\item
  \textbf{Dubitativas}. Expresan una duda.
\item
  \textbf{Desiderativas}. Expresan un deseo o un ruego.
\end{itemize}

\end{block}

\end{frame}

\hypertarget{actividades}{%
\section{Actividades}\label{actividades}}

\begin{frame}{Actividades}
\protect\hypertarget{actividades-1}{}

\begin{itemize}
\item
  \begin{enumerate}
  \tightlist
  \item
    Inventa una oración para cada uno de los siguientes tipos:
  \end{enumerate}

  \begin{itemize}
  \tightlist
  \item
    Exhortativas
  \item
    Enunciativas (afirmativa)
  \item
    Enunciativa (negativa)
  \item
    Interrogativas (directa)
  \item
    Interrogativa (indirecta)
  \item
    Desiderativas
  \item
    Dubitativas
  \item
    Exclamativas
  \end{itemize}
\item
  \begin{enumerate}
  \setcounter{enumi}{1}
  \tightlist
  \item
    Convertir interrogativas directas en indirectas:
  \end{enumerate}

  \begin{itemize}
  \tightlist
  \item
    ¿Cuántos hermanos tienes?
  \item
    ¿Cuál es tu color favorito?
  \item
    ¿Han venido tus tías por tu cumpleaños?
  \item
    ¿Vas a ira al dice hoy?
  \end{itemize}
\end{itemize}

\end{frame}

\hypertarget{examen}{%
\section{Examen}\label{examen}}

\begin{frame}{Posibles preguntas de examen}
\protect\hypertarget{posibles-preguntas-de-examen}{}

\begin{enumerate}
\tightlist
\item
  Hacer una lista con los distintos tipos de oraciones
\item
  Clasificar oraciones de un texto
\end{enumerate}

\end{frame}

\end{document}
